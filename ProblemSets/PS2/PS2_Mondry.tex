\documentclass{article}

% Language setting
% Replace `english' with e.g. `spanish' to change the document language
\usepackage[english]{babel}

% Set page size and margins
% Replace `letterpaper' with`a4paper' for UK/EU standard size
\usepackage[letterpaper,top=2cm,bottom=2cm,left=3cm,right=3cm,marginparwidth=1.75cm]{geometry}

% Useful packages
\usepackage{amsmath}
\usepackage{graphicx}
\usepackage[colorlinks=true, allcolors=blue]{hyperref}

\title{Problem Set 2 - ECON 5253}
\author{Thomas Mondry}

\begin{document}
\maketitle

\section{The data scientist's tools}

I understand data science as a process for using data to extract insights; the tools used by data scientists can be organized by the role they play in this process.

\begin{itemize}
    \item Tools for working with data
    \begin{itemize}
        \item Data acquisition: Web scraping, usually either through manual HTML scraping or APIs
        \item Big data management: File splitting (useful in limited situations), computing clusters \& RDDs (using software such as Hadoop or Spark), database transformation tools (i.e. SQL)
    \end{itemize}
    \item Tools for getting insights from data
    \begin{itemize}
        \item Data visualization: Packages built to rurn natively in programming languages (e.g. R's ggplot2, Python's matplotlib, Julia's Plots.jl), or third party visualization software (e.g. Tableau, Power BI)
        \item Project formulation \& measurement: awareness of what is contained in the data \& quantifiable parameter(s) of interest
        \item Modeling:
        \begin{itemize}
            \item Goals: testing theories, predicting behavior, extracting causality (especially useful in econometric applications)
            \item Programming languages: R and Python (most common), Julia (relatively new \& powerful)
            \item Model types: regression, classification, clustering
        \end{itemize}
    \end{itemize}
\end{itemize}

\end{document}