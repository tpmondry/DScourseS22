\documentclass{article}

% Language setting
% Replace `english' with e.g. `spanish' to change the document language
\usepackage[english]{babel}

% Set page size and margins
% Replace `letterpaper' with`a4paper' for UK/EU standard size
\usepackage[letterpaper,top=2cm,bottom=2cm,left=3cm,right=3cm,marginparwidth=1.75cm]{geometry}

% Useful packages
\usepackage{amsmath}
\usepackage{graphicx}
\usepackage[colorlinks=true, allcolors=blue]{hyperref}

\title{Problem Set 1 - ECON 5253}
\author{Thomas Mondry}

\begin{document}
\maketitle

\section{Introduction}

I found my way from a math major into economics after finding "advanced" (or generally analysis-based) mathematics courses less interesting because of the lack of focus on applications; economics constituted a highly ordered, quantitative approach to describing real-world phenomena using sophisticated mathematical techniques, which checked all the boxes. I've never considered economics to be a passion of mine -- I'm not very interested in policy, but I've always enjoyed "thinking like an economist," particularly with applied microeconomics. I've also found an interest in behavioral \& experimental economics, and I am working on my MA research project in this area this semester.

I am very interested in data science as a tool used to extract meaningful insights from data. I feel that I have a sufficiently strong background in statistical theory that allows me to understand the strengths, weaknesses and appropriateness of various data science techniques with regard to solving different real-world problems; this is the skill that I am most focused on improving as I continue to study data science, because almost everything else that analytics professionals do is trivial given the strength of statistical packages. I took a data science course last semester which was mainly focused on providing a broad survey of modeling techniques (both regression and classification). I'm especially interested in this class because it was designed for economists/applications in econometrics, which implies a focus on causality. While, in many contexts, predictive power may be the most important consideration in model selection, I don't believe it should often be the only consideration.

I'd like to use this class to develop a stronger repertoire of general computing skills which extend to earlier parts of the data science process (outside of getting from a semi-clean CSV to results, which much of my coursework has focused on in the past). I'm fairly familiar with R after a couple of years of regular practice in my coursework and SQL from work, so I also hope to get familiar with Python and exposed to Julia through this course. I don't currently have any ideas for the final project, and I hope to be struck by inspiration after getting a better look at the topics we'll be covering in the course.

After graduation this semester with my BA/MA in managerial economics, I will be working as an actuarial analyst in a rotational actuarial/analytics program at Travelers in St. Paul, MN.

\section{Equation}

\[a^2+b^2=c^2\]

\end{document}